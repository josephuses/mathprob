\chapter{Integrals over Subsets}

\begin{lemma}
Let $T$ be a measurable map from $(\Omega_1, \F_1, \mu)$ to $(\Omega, \F_2)$. For $A\in\F_2$ define
\begin{equation*}
(\mu\circ T^{-1})(A)=\mu(T^{-1}(A)).
\end{equation*}
Then $\mu\circ T^{-1}$ is a measure on $(\Omega_2, \F_2)$.
\end{lemma}



\begin{proof}
For disjoint elements $A_1$, $A_2$, \ldots of $\F_2$ we have
\begin{align*}
(\mu\circ T^{-1})\left(\bigcup_n A_n \right) & = \mu\left(T^{-1}\left(\bigcup_n A_n\right)\right)\\
 &= \mu\left(\bigcup_n T^{-1}(A_n) \right)\\
 &= \sum_n \mu(T^{-1}(A_n))\\
 &=\sum_n (\mu\circ T^{-1})(A_n).
\end{align*}
\end{proof}




\begin{theorem}[Change of Variable]
Let $T$ be a measurable map from $(\Omega_1, \F_1, \mu)$ to $(\Omega_2, \F_2)$. Let $f$ be a measurable function on $\Omega_2$. Then
\begin{equation*}
\int fd(\mu\circ T^{-1})=\int f\circ T d\mu
\end{equation*}
if either integral is defined (possibly infinite).
\end{theorem}




\begin{proof}
For $f=I_A$, note that $f\circ T=I_{T^{-1}(A)}$ and, hence, 
\begin{equation*}
\int fd(\mu\circ T^{-1})=(\mu\circ T^{-1})(A)=\mu(T^{-1}(A))=\int f\circ Td\mu.
\end{equation*}
By linearity of the integral, the result is then true for simple function $f$. Let $f$ be a nonnegative measurable function and $g_1$, $g_2$, \ldots\ be an increasing sequence of simple functions with limit $f$. Then by MCT
\begin{align*}
\int fd(\mu\circ T^{-1})&=\int f^+d(\mu\circ T^{-1})-\int f^{-}d(\mu\circ T^{-1})\\
 &=\int f^+\circ Td\mu - \int f^-\circ Td\mu\\
 &=f\circ T d\mu.
\end{align*}
\end{proof}


\begin{definition}[Lebesque integral]
The integral of $f$ over a set $A$ in $\F$ is defined by
\begin{equation*}
\int_A fd\mu = \int fI_A d\mu
\end{equation*}
If $\lambda$ is Lebesque measure then the integral $\int_{[a,b]}f d\lambda$ is called the Lebesque integral of $f$ and is usually denoted by $\int_{a}^{b} f(x)dx$.
\end{definition}



\begin{theorem}
Let $f$ be a bounded function defined on the closed, bounded interval $[a,b]$. If $f$ is Riemann integrable over $[a,b]$, then it is Lebesque integrable over $[a,b]$ and the two integrals are equal.
\end{theorem}



\begin{definition}[Lebesque-Stieltjes integral]
Let $F$ be a nondecreasing and right-continuous function on $\Re$. Define $\mu_F((x,y])=F(y)-F(x)$ for $x\leq y$. Then the integral $\int_{[a,b]}f d\mu_F$ is called the Lebesque-Stieltjes integral of $f$ with respect to $F$ and is usually denoted by $\int_{a}^{b}f(x)dF(x)$.
\end{definition}



\begin{lemma}
If $A_1$, $A_2$, \ldots are disjoint, and if $f$ is either nonnegative or integrable, then
\begin{equation*}
\int_{\bigcup_n A_n} fd\mu = \sum_n \int_{A_n} f d\mu.
\end{equation*}
\end{lemma}




\begin{proof}
\begin{align*}
\int_{\bigcup_n A_n} f d\mu &= \int fI_{\bigcup_n A_n} d\mu\\
	&=\lim\limits_{N\to\infty}\inf fI_{\bigcup_{n=1}^N A_n}d\mu\\
	&=\lim\limits_{N\to\infty}\sum_{n=1}^{N}\int f I_{A_n}d\mu\\
	&=\sum_n\int_{A_n} f d\mu.
\end{align*}
\end{proof}



\begin{theorem}
Let $f$ be a nonnegative measurable function. Define the set function $\nu$ on $\F$ by
\begin{equation*}
\nu(A)=\int_A f d\mu.
\end{equation*}
Then $\nu$ is a measure.
\end{theorem}




\begin{proof}
From the lemma
\begin{equation*}
\nu\left(\bigcup A_n\right)=\sum \int_{A_n} f d\mu = \sum \nu(A_n)
\end{equation*}
for disjoint $A_1$, $A_2$, \ldots in $\F$.
\end{proof}



\begin{remark}
We say that the measure $\nu$ has density $f$ relative to $\mu$, and write $d\nu/d\mu=f$. Moreover, note that for $E$ in $\F$, $\mu(E)=0$ implies that $\nu(E)=0$.
\end{remark}




\begin{theorem}[Chain Rule]
Let $f$ be a nonnegative measurable function and suppose that $f=d\nu/d\mu$. If 
$g$ is a measurable function. Then
\begin{equation*}
\int gd\nu = \int gfd\mu.
\end{equation*}
whenever either side exists.
\end{theorem}



\begin{proof}
For $g=I_A$, we have
\begin{equation*}
\int gd\nu=\nu(A)=\int_A fd\mu = \int gfd\mu.
\end{equation*}
By linearity of integrals, it follows that the result is true for simple functions.
That result follows for nonnegative unctions now follows from MCT the fact that any 
nonnegative function is an increasing limit of simple functions. finally, the result follows for integrable $f$ by applying the equation to $f^+$ and $f^-$, separately.
\end{proof}