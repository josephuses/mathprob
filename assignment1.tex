\documentclass{article}

\usepackage[%
inner=1.5in,
outer=1in,
top=1in,
bottom=1in
]{geometry}

\usepackage{amsmath, amssymb, amsthm}
\newtheorem{definition}{Definition}[section]
\newtheorem{example}{Example}[section]
\newtheorem{theorem}{Theorem}
\newtheorem*{theorem*}{Theorem}
\newtheorem{notes}{Note(s)}

\usepackage{mathrsfs}

\usepackage{enumitem}
\newlist{prooflist}{enumerate}{1}
\setlist[prooflist]{label=(\emph{\roman*})}

\usepackage{microtype}
\usepackage[T1]{fontenc}
\usepackage[utf8]{inputenc}

\newcommand{\D}{\ensuremath{\mathcal D}}
\newcommand{\F}{\ensuremath{\mathcal F}}
\newcommand{\J}{\ensuremath{\mathcal J}}
\newcommand{\G}{\ensuremath{\mathcal G}}

\title{Stat 235 Exercise 1}
\author{Joseph S. Tabadero, Jr.}
\date{February 19, 2016}

\begin{document}
	
	\maketitle

\begin{theorem*}[$\star$]\label{thm:3}
If $\mathcal J$ is a $\pi$-system and $\mathcal D$ is a $d$-system with $\mathcal J\subset \mathcal D$, then $\sigma(\mathcal J)\subset \mathcal D$.
\end{theorem*}

\begin{notes}
$\sigma(\mathcal J)\subset \mathcal D$ means that $\D$ contains the smallest $\sigma$-algebra contained in $\J$. 
\end{notes}

\begin{proof}
To show $\sigma(\J)\subset \D$, we need only show that $\D$ is a $\pi$-system. Why? If $\D$ is a $\pi$-system, then, since it is also a $d$-system, then it is a $\sigma$-algebra. $\sigma(\J)$ and $\D$ both being $\sigma$-algebras and $\sigma(\J)$ being the smallest $\sigma$-algebra containing $\J$ means $\sigma(\J)\subset \D$.

Now without loss of generality, we can also let $\D$ be the smallest such $d$-system that contains $\J$. If it holds that $\sigma(\J)\subset \D$ for this $\D$ then it will hold for other bigger $\D$'s.

Consider $\F=\{A\in\D\mid A\cap B\in\D\; \text{for all}\; B\in\J \}$. We need to show that $\J\subset \F$ and $\F$ is a $d$-system.

Note than since $\J$ is a $\pi$-system, for $A\in\J$ then $A\cap B\in \D$ for all $B\in\J$. Hence, $A\in\F$ and $\J\subset\F$.


We now proceed to show that $\F$ is a $d$-system. 
\begin{prooflist}
	\item Now $\Omega\cap B=B\in\J$ for all $B\in\J$ and hence $\Omega\in\F$.
	\item Moreover, if $A,C\in\F$ and $A\subset C$ then for any $B\in \J$, $(C\setminus A)\cap B=(C\cap B)\setminus (A\cap B)$. Now $C\cap B\in \D$ and $A\cap B \D$ by definition of $\F$. So $(C\setminus A)\cap B\in \D$ since $\D$ is a $d$-system. Thus $C\setminus A$ satisfies the condition of $\F$ so $C\setminus A\in \F$.
	\item $A_1, A_2, \ldots \in\F$ and $A_1\subset A_2\subset\cdots$. Then for any $B\in\J$, $\left(\bigcup_{i=1}^\infty A_i\right)\cap B = \bigcup_{i=1}^\infty(A_i\cap B)$. Now $A_i\cap B\in\D$ for all $i$ since $A_i\in\F$ and $(A_1\cap B)\subset (A_2\cap B)\subset (A_3\cap B)\subset \cdots$. This means that $\left(\bigcup_{i=1}^\infty A_i\right)\cap B=\bigcup_{i=1}^\infty(A_i\cap B)\in\D$, since $\D$ is a $d$-system. Therefore, $\left(\bigcup_{i=1}^\infty A_i\right)\in\F$. Therefore, $\F$ is a $d$-system.
\end{prooflist}



Now since $\D$ is the smallest $d$-system containing $\J$, and $\J\subset\D$ it follows that $\F=\D$. Therefore, if $A\in\D$ and $B\in\J$, then $A\cap B\in\D$.

Now define $\G=\{B\in\D\mid A\cap B\in\D\; \text{for all}\; A\in\D  \}$. Note that $\J\subset\G$. We can show in a similar fashion that $\G$ is a $d$-system (see Exercise \ref{exer:1}). Hence, $\G=\D$. Therefore, if $A,B\in\D$ then $A\cap B\in\D$ and hence $\D$ is a $\pi$-system. Now $\D$ is then both a $\pi$-system and a $d$-system, and, whence, a $\sigma$-algebra. Therefore, $\sigma(\J)\subset\D$.
\end{proof}

\section*{Exercises}

\begin{enumerate}
\item\label{exer:1} Show that $\G=\{B\in\D\mid A\cap B\in\D\; \text{for all}\; A\in\D  \}$ in the proof of Theorem $(\star)$ is a $d$-system.

\begin{proof}
\begin{prooflist}
	\item Now $\Omega\in\D$ since $\D$ is a $d$-system. Let $A\in\D$. Now, $\Omega\cap A=A\in\D$. Therefore, $\Omega\in\G$.
	\item Moreover, if $B,C\in\G$ and $B\subset C$ then for any $A\in \D$, $(C\setminus B)\cap A=(A\cap C)\setminus (A\cap B)$. Now $A\cap C\in\D$ and $A\cap B\in \D$ by definition of $\G$. So $(C\setminus A)\cap B\in \D$ since $\D$ is a $d$-system. Thus $C\setminus A$ satisfies the condition of $\G$ so $C\setminus A\in \G$.
	\item Suppose $B_1, B_2, \ldots \in\G$ and $B_1\subset B_2\subset\cdots$. Then for any arbitrary $A\in\D$, $\left(\bigcup_{i=1}^\infty B_i\right)\cap A = \bigcup_{i=1}^\infty(B_i\cap A)$. Now $B_i\cap A\in\D$ for all $i$ since $B_i\in\G$ and $(B_1\cap A)\subset (B_2\cap A)\subset (B_3\cap A)\subset \cdots$. This means that $\left(\bigcup_{i=1}^\infty B_i\right)\cap A=\bigcup_{i=1}^\infty(B_i\cap A)\in\D$, since $\D$ is a $d$-system. Therefore, $\left(\bigcup_{i=1}^\infty B_i\right)\in\G$. Therefore, $\G$ is a $d$-system.
\end{prooflist}
\end{proof}

\end{enumerate}


\end{document}