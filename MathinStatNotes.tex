\documentclass{amsbook}

\usepackage[%
inner=1.5in,
outer=1in,
top=1in,
bottom=1in
]{geometry}

\usepackage{graphicx}
\usepackage{hyperref}
\usepackage{floatrow}
\floatsetup[table]{capposition=top}
\floatsetup[table]{capposition=bottom}


\usepackage{tikz}
\usetikzlibrary{arrows}

\usepackage{amsmath, amssymb, amsthm}
\newcommand{\redover}[2]{\overset{\textcolor{red}{#1}}{#2}}


\newtheorem{definition}{Definition}[chapter]
\newtheorem{example}{Example}[chapter]
\newtheorem{theorem}{Theorem}[chapter]
\newtheorem{notes}{Remarks}[chapter]
\newtheorem{corollary}{Corollary}[chapter]
\newtheorem{lemma}{Lemma}[theorem]
\newtheorem{remark}{Remark}[theorem]

\usepackage{Alegreya}
\usepackage{venndiagram}
\usetikzlibrary{arrows}
\definecolor{qqqqff}{rgb}{0.3333333333333333,0.3333333333333333,0.3333333333333333}

\usepackage{mathrsfs}

\usepackage{enumitem}
\newlist{prooflist}{enumerate}{1}
\setlist[prooflist]{label=(\emph{\roman*})}

\usepackage{microtype}
\usepackage[T1]{fontenc}
\usepackage[utf8]{inputenc}

\newcommand{\B}{\ensuremath{\mathcal B}}
\newcommand{\D}{\ensuremath{\mathcal D}}
\newcommand{\F}{\ensuremath{\mathcal F}}
\newcommand{\J}{\ensuremath{\mathcal J}}
\newcommand{\G}{\ensuremath{\mathcal G}}
\newcommand{\Prob}{\ensuremath{\mathbb P}}

\newcommand{\abs}[1]{\left\lvert #1\right\rvert}
\newcommand{\prob}{\mathbb{P}}
\newcommand{\Lp}{\ensuremath{\mathcal L}}
\newcommand{\norm}[1]{\left\lVert#1\right\rVert}

\title{Notes in Stat 235\\
	Mathematics in Statistics\\
	from Prof. Dorado's Lecture
	}
\author{Joseph S. Tabadero, Jr.}
\date{2nd Semester, AY 2015-2016}

\begin{document}

\maketitle
\tableofcontents

\chapter*{Notations}
\addcontentsline{toc}{chapter}{Notations}

\begin{tabular}{ll}
$\subset $ & subset\\
$\forall$ & for all, for every\\
$\exists$ & there exists, there exists at least one, for some\\
$\ni$ & such that\\
$\cup$ & union\\
$\cap$ & intersection\\
$\setminus$ & set difference operator\\
$\qedsymbol$ & Q.E.D, end of proof\\
$\F$ & A $\sigma$-algebra, unless stated otherwise\\
\end{tabular}

\input intro


\input measures


\input continuity-and-uniqueness

\input caratheodory

\input measurable-functions

\input prob-spaces

\input construction-int

\input limit-thm-int

\input int-over-subsets

\input product-measure

\input lp-spaces

\end{document}